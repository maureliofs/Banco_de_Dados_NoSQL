\documentclass[glossy]{beamer}
\useoutertheme{wuerzburg}
\useinnertheme[realshadow,corners=2pt,padding=2pt]{chamfered}
\usecolortheme{shark}
\usepackage[utf8]{inputenc} % codificacao de caracteres
\usepackage[T1]{fontenc}    % codificacao de fontes
\usepackage[brazil]{babel}
\usepackage{graphicx} % Allows including images
\usepackage{booktabs} % Allows the use of \toprule, \midrule and \bottomrule in tables
\usepackage[absolute,overlay]{textpos}
\usepackage{movie15}

\usepackage{tikz}
\newcommand<>{\hover}[1]{\uncover#2{%
 \begin{tikzpicture}[remember picture,overlay]%
 \draw[fill,opacity=0.4] (current page.south west)
 rectangle (current page.north east);
 \node at (current page.center) {#1};
 \end{tikzpicture}}
}

\title{Banco de Dados NoSQL}
\author{Igor Moraes, Lucas Fonseca, Marco Aurélio}
\institute{Universidade Federal de Lavras}
\date{5 de outubro de 2018}

\begin{document}

\begin{frame}
\maketitle
\end{frame}

\begin{frame}
\frametitle{Índice} % Table of contents slide, comment this block out to remove it
\tableofcontents % Throughout your presentation, if you choose to use \section{} and \subsection{} commands, these will automatically be printed on this slide as an overview of your presentation
\begin{textblock*}{5cm}(9cm,7cm) % {block width} (coords)
\includegraphics[scale=0.3]{no-sql.png}
\end{textblock*}

\end{frame}

\section{Introdução Banco de Dados NoSQL}
\section{Redes Sociais e Bancos de Dados NoSQL}
\section{Características dos Banco de Dados NoSQL}
\section{Por que usar Banco de Dados NoSQL?}
\section{Modelos de Dados:}
\subsection{Chave-Valor}
\subsubsection{Redis}
\subsection{Documento}
\subsubsection{MongoDB}
\subsection{Orientado à Coluna}
\subsubsection{Cassandra}
\subsection{Grafo}
\subsubsection{Neo4J}
\section{Comparativo dos Principais SGBD’s NoSQL}
\section{Conclusão}
\section{}

\begin{frame}
\frametitle{Introdução Banco de Dados NoSQL}
\begin{itemize}
    \item O termo foi usado inicialmente como nome de um banco de dados não relacional de código aberto.
    \item É completamente distinto do modelo relacional e portanto deveria ser mais apropriadamente chamado "NoREL" ou algo que produzisse o mesmo efeito. 
    \item Referência ao bancos de dados relacionais mais populares MySQL, Microsoft SQL Server, PostgreSQL.
\end{itemize}
\begin{figure}[h]
    \centering
    \includegraphics[scale=0.5]{nosql-1t.png}
\end{figure} 
\end{frame}

\begin{frame}{Redes Sociais e Bancos de Dados NoSQL}
\begin{figure}[h]
    \centering
    \includegraphics[scale=0.35]{redes.png}
\end{figure}    
\end{frame}

\begin{frame}
\frametitle{Características dos Banco de Dados NoSQL}
\begin{itemize}
    \item Utilização do processamento paralelo para processamento das informações.
    \item Distribuição em escala global.
    \item Diversos tipos para diferentes aplicações.
\end{itemize}
\begin{figure}[h]
    \centering
    \includegraphics[scale=0.35]{NoSQLS.png}
\end{figure} 
\end{frame}


\begin{frame}
\frametitle{Por que usar Banco de Dados NoSQL?}
\begin{itemize}
    \item Flexibilidade
    \item Escalabilidade
    \item Disponibilidade
    \item Raízes Open Source
    \item Baixo Custo Operacional
    \item Funcionalidades Especiais
\end{itemize}

\begin{textblock*}{5cm}(7cm,2cm) % {block width} (coords)
    \includegraphics[scale=0.18]{scrum-logo.png}
\end{textblock*}

\begin{textblock*}{5cm}(7cm,4cm) % {block width} (coords)
    \includegraphics[scale=0.22]{cluster.png}
\end{textblock*} 

\begin{textblock*}{5cm}(9.5cm,4cm) % {block width} (coords)
    \includegraphics[scale=0.18]{247.png}
\end{textblock*} 

\begin{textblock*}{5cm}(6.7cm,6.2cm) % {block width} (coords)
    \includegraphics[scale=0.14]{os.png}
\end{textblock*}

\begin{textblock*}{2cm}(9.5cm,6.2cm) % {block width} (coords)
    1\includegraphics[scale=0.14]{lc.png}
\end{textblock*}

\end{frame}

\begin{frame}
\frametitle{Modelos de Dados NoSQL}
\begin{figure}[h]
    \centering
    \includegraphics[scale=0.69]{types.png}
\end{figure}
\end{frame}

\begin{frame}
\frametitle{Principais SGBD's do mercado devido à popularidade}
\begin{figure}[h]
    \centering
    \includegraphics[scale=0.3]{ranking.png}
\end{figure}
\end{frame}

\begin{frame}
\frametitle{Chave-Valor}
\begin{itemize}
    \item Armazena dados como um conjunto de pares de chave-valor.
    \item A chave funciona como um identificador exclusivo.
    \item A chave e os valores podem ser desde objetos simples até objetos compostos complexos.
    \item Principal SGBD: Redis.
\end{itemize}
\begin{textblock*}{5cm}(7.5cm,6cm) % {block width} (coords)
    \includegraphics[scale=0.2]{key2.png}
\end{textblock*}
\end{frame}

\begin{frame}
\frametitle{Redis}
\begin{itemize}
    \item \textbf{RE}mote \textbf{DI}ctionary \textbf{S}erver
    \item Criado por  Salvatore Sanfiippo;
    \item Escrito na Linguagem C;
    \item Open-source (Licença BSD);
    \item Ideia de Hash/Dicionário em LP's
    \item Dados armazenados na memória;
    \item Comandos atômicos;
    \item Single-threaded;
    \item Modelo cliente-servidor (TCP);
    \item Alta performance para gravação e/ou leitura de dados 
    \item Interação através de comandos, não há linguagem de consulta semelhante ao SQL;
\end{itemize}
\begin{textblock*}{5cm}(7cm,2cm) % {block width} (coords)
    \includegraphics[scale=0.4]{redis.png}
\end{textblock*}
\end{frame}

\begin{frame}{Exemplo Redis}
    \begin{figure}[h]
        \centering
        \includegraphics[scale=0.7]{r1.png}
        \caption{Comandos básicos do Redis}
    \end{figure}
\end{frame}

\begin{frame}{Exemplo Redis}
    \begin{figure}[h]
        \centering
        \includegraphics[scale=0.7]{r2.png}
        \caption{Um valor de Redis pode ser configurado para expirar}
    \end{figure}
\end{frame}

\begin{frame}{Exemplo Redis}
    \begin{figure}[h]
        \centering
        \includegraphics[scale=0.7]{r3.png}
        \caption{Listas do Redis}
    \end{figure}
\end{frame}


\begin{frame}{Aplicações do Redis}
\begin{itemize}
    \item Aplicações Web e móveis;
    \item Jogos;
    \item Tecnologia de anúncios;
    \item IOT;
\end{itemize} 
Empresas que usam Redis:
\begin{itemize}
    \item Twitter;
    \item GitHub;
    \item Pinterest;
    \item Snapchat;
    \item StackOverflow;
    \item Flickr;
\end{itemize}
\end{frame}

\begin{frame}
\frametitle{Documento}
\begin{itemize}
    \item Armazena os dados semiestruturados como documentos;
    \item Intuitivo para desenvolvedores por se assimilar ao JSON/XML;
    \item Cada documento é autodescritível;
    \item Os documentos são agrupados em “conjuntos”;
    \item Identificadores únicos universais (UUID);
    \item Possibilita a consulta de documentos através de métodos avançados de agrupamento e filtragem (MapReduce);
\end{itemize}
\begin{textblock*}{5cm}(10cm,6.5cm) % {block width} (coords)
    \includegraphics[scale=0.55]{doc.png}
\end{textblock*}
\end{frame}

\begin{frame}
\frametitle{MongoDB}
\begin{itemize}
    \item Escrito na linguagem C++;
    \item Open-source (Licença GNU AGPL 3.0);
    \item Armazena dados em documentos flexíveis semelhantes a JSON.
    \item O modelo de documento é mapeado para os objetos no código do seu aplicativo , facilitando o trabalho com os dados;
    \item Distribuído em seu núcleo, de modo que a alta disponibilidade, o dimensionamento horizontal e a distribuição geográfica são integrados;
    \end{itemize}
\begin{textblock*}{5cm}(6cm,1.5cm) % {block width} (coords)
    \includegraphics[scale=0.45]{mongo1.png}
\end{textblock*}
\end{frame}

\begin{frame}{Exemplo MongoDB}
    \begin{figure}[h]
        \centering
        \includegraphics[scale=0.7]{m1.png}
        \caption{Inserindo os dados}
    \end{figure}
\end{frame}

\begin{frame}{Exemplo MongoDB}
    \begin{figure}[h]
        \centering
        \includegraphics[scale=0.7]{m3.png}
        \caption{Salvando os dados}
    \end{figure}
\end{frame}

\begin{frame}{Exemplo MongoDB}
    \begin{figure}[h]
        \centering
        \includegraphics[scale=0.7]{m4.png}
        \caption{Conteúdo de MeusDados}
    \end{figure}
\end{frame}

\begin{frame}{Exemplo MongoDB}
    \begin{figure}[h]
        \centering
        \includegraphics[scale=0.55]{m5.png}
        \caption{Conteúdo de MeusDados em outra armazenagem}
    \end{figure}
\end{frame}

\begin{frame}{Exemplo MongoDB}
    \begin{figure}[h]
        \centering
        \includegraphics[scale=0.7]{m7.png}
        \caption{Removendo dados}
    \end{figure}
\end{frame}

\begin{frame}{Exemplo MongoDB}
    \begin{figure}[h]
        \centering
        \includegraphics[scale=0.6]{m8.png}
        \caption{Atualizando dados}
    \end{figure}
\end{frame}

\begin{frame}{Exemplo MongoDB}
    \begin{figure}[h]
        \centering
        \includegraphics[scale=0.7]{m9.png}
        \caption{Consultando dados}
    \end{figure}
\end{frame}

\begin{frame}{Aplicações do MongoDB}
\begin{itemize}
    \item Aplicativos de gerenciamento de conteúdo, como blogs e plataformas de vídeo;
    \item Armazenamento de informações de catálogo, por exemplo: Comércio eletrônico;
    \item Aplicações em que precisa-se gerencia diversos atributos sem afetar outros;
\end{itemize}
Empresas que usam MongoDB:
\begin{itemize}
    \item Globo.com;
    \item SourceForge;
    \item FourSquare;
    \item MailBox(e-mail do Dropbox);
    \item LinkedIn;
\end{itemize}
\end{frame}

\begin{frame}
\frametitle{Orientado à Coluna}
\begin{itemize}
    \item Criado para trabalhar com grande quantidade de dados;
    \item Trabalha apenas com colunas, cada coluna é um arquivo diferente no disco;
    \item Buscar por dados é muito mais rápido quando se tem uma grande quantidade de colunas na tabela;
    \item Compressão de dados maior do que em linha;
    \item Busca é feita antes da criação das tuplas;
\end{itemize}
\begin{textblock*}{5cm}(9.5cm,6.4cm) % {block width} (coords)
    \includegraphics[scale=0.06]{coluna.png}
\end{textblock*}
\end{frame}

\begin{frame}{Linha vs. Coluna}
\begin{textblock*}{5cm}(0.5cm,1.5cm) % {block width} (coords)
    \includegraphics[scale=0.6]{linhavscoluna1.jpg}
\end{textblock*}
\begin{textblock*}{5cm}(6cm,5.5cm) % {block width} (coords)
    \includegraphics[scale=0.6]{linhavscoluna2.png}
\end{textblock*}    
\end{frame}

\begin{frame}
\frametitle{Cassandra}
\begin{itemize}
    \item Inicialmente criado no facebook e hoje é mantido pela Apache;
    \item Velocidade de gravação de até 360MB/s, mais rápido que as buscas;
    \item Armazenamento distribuído por hash em nós;
    \item Nós são replicados;
    \item Linguagem: CQL (Cassandra Query Language)
\end{itemize}
\begin{textblock*}{5cm}(8cm,5.5cm) % {block width} (coords)
    \includegraphics[scale=0.06]{cassandra.png}
\end{textblock*}
\end{frame}

\begin{frame}{Arquitetura do Cassandra}
\begin{figure}[h]
    \centering
    \includegraphics[scale=0.6]{grafo.png}
\end{figure} 
\end{frame}

\begin{frame}{Exemplo Cassandra}
    \begin{figure}[h]
        \centering
        \includegraphics[scale=0.7]{c1.png}
        \caption{Comando CREATE TABLE}
    \end{figure}
\end{frame}

\begin{frame}{Exemplo Cassandra}
    \begin{figure}[h]
        \centering
        \includegraphics[scale=0.55]{c2.png}
        \caption{Comando INSERT}
    \end{figure}
\end{frame}

\begin{frame}{Exemplo Cassandra}
    \begin{figure}[h]
        \centering
        \includegraphics[scale=0.52]{c3.png}
        \caption{Comando SELECT}
    \end{figure}
\end{frame}

\begin{frame}{Exemplo Cassandra}
    \begin{figure}[h]
        \centering
        \includegraphics[scale=0.50]{c4.png}
        \caption{SELECT sem chave na cláusula WHERE}
    \end{figure}
\end{frame}



\begin{frame}{Aplicações do Cassandra}
\begin{itemize}
    \item Catálogo de produtos;
    \item Redes sociais; 
    \item Detecção de fraudes;
    \item Aplicações analíticas;
\end{itemize}
Empresas que usam Cassandra:
\begin{itemize}
    \item Amazon;
    \item eBay;
    \item Netflix;
    \item Facebook;
    \item Microsoft;
    \item Instagram;
    \item NASA;
\end{itemize}
\end{frame}

\begin{frame}
\frametitle{Grafo}
\begin{itemize}
\item Motivado pela grande quantidade de dados sendo geradas diariamente (Big Data);
\item Expressa de forma explícita os relacionamentos através de:
    \begin{itemize}
        \item Relacionamentos = arestas;
        \item Entidades = Vértices;
    \end{itemize}
\item Modelagem mais simples, linguagem mais natural;
\item Usado para consultas complexas;
\item Relacionamentos podem possuir dados;
\end{itemize}
\begin{textblock*}{5cm}(8.7cm,5.7cm) % {block width} (coords)
    \includegraphics[scale=0.5]{grafoex.png}
\end{textblock*}
\end{frame}

\begin{frame}{Exemplo Twitter}
\begin{figure}[h]
    \centering
    \includegraphics[scale=0.65]{ex.png}
\end{figure}
\end{frame}

\begin{frame}
\frametitle{Neo4J}
\begin{itemize}
\item Desenvolvido em Java;
\item Banco de dados em grafo mais popular; 
\item Open Source;
\item Linguagem: Cypher
\end{itemize}
\begin{textblock*}{5cm}(8cm,5.7cm) % {block width} (coords)
    \includegraphics[scale=0.08]{neo4j.png}
\end{textblock*}
\end{frame}

\begin{frame}{Exemplo Neo4J}
    \begin{figure}[h]
        \centering
        \includegraphics[scale=0.29]{g1.jpg}
        \caption{Criação de bandas e fãs (Cypher)}
    \end{figure}
\end{frame}

\begin{frame}{Exemplo Neo4J}
    \begin{figure}[h]
        \centering
        \includegraphics[scale=0.16]{g2.jpg}
        \caption{Grafo}
    \end{figure}
\end{frame}

\begin{frame}{Exemplo Neo4J}
    \begin{figure}[h]
        \centering
        \includegraphics[scale=0.27]{g3.jpg}
        \caption{Criação dos relacionamentos entre fãs e bandas (Cypher)}
    \end{figure}
\end{frame}

\begin{frame}{Exemplo Neo4J}
    \begin{figure}[h]
        \centering
        \includegraphics[scale=0.15]{g4.jpg}
        \caption{Grafo}
    \end{figure}
\end{frame}

\begin{frame}{Exemplo Neo4J}
    \begin{figure}[h]
        \centering
        \includegraphics[scale=0.15]{g5.jpg}
        \caption{Quais são as bandas que Eder curte? (Cypher)}
    \end{figure}
\end{frame}

\begin{frame}{Exemplo Neo4J}
    \begin{figure}[h]
        \centering
        \includegraphics[scale=0.16]{g6.jpg}
        \caption{Grafo}
    \end{figure}
\end{frame}

\begin{frame}{Aplicações do Neo4J}
\begin{itemize}
    \item Detecção de fraude e solução de análise;
    \item Ferramenta de recomendação e recomendação de produto;
    \item Mídias sociais e Redes Sociais
    \item Inteligência artificial e Machine Learning;
\end{itemize}
Empresas que usam Neo4J:
\begin{itemize}
    \item eBay;
    \item Walmart;
    \item Medium
    \item Linkedin;
    \item Gamesys;
\end{itemize}
\end{frame}


\begin{frame}
\frametitle{Comparativo dos Principais SGBD's NoSQL}
\begin{figure}[h]
\centering
\includegraphics[scale=0.36]{grafic.png}
\end{figure}
\end{frame}

\begin{frame}
\frametitle{Conclusão}
Com o grande crescimento do volume de dados em determinadas organizações, os bancos de dados NoSQL tem se tornado uma grande alternativa quando nos referimos a escalabilidade e disponibilidade, fatores estes que se tornam imprescindíveis em algumas aplicações Web.
\end{frame}


\begin{frame}
\begin{figure}[h]
\centering
\includegraphics[scale=0.14]{duvidas.png}
\end{figure}
\end{frame}

\begin{frame}
\frametitle{Referências Bibliográficas}
\begin{itemize}
\item https://imasters.com.br/banco-de-dados/introducao-ao-redis-o-nosql-chave-valor-mais-famoso
\item http://desenvolvedor.ninja/redis-o-que-e-e-para-que-serve/
\item https://aws.amazon.com/pt/elasticache/what-is-redis/
\item http://blog.clubedocodigo.com.br/crie-aplicacoes-mais-rapidas-utilizando-redis/
\item https://diogobemfica.com.br/comecando-com-o-redis/
\item https://www.ibm.com/developerworks/br/library/j-javadev2-22/index.html
\item https://aws.amazon.com/pt/nosql/document/
\end{itemize}
\end{frame}

\begin{frame}
\frametitle{Referências Bibliográficas}
\begin{itemize}
\item https://www.devmedia.com.br/introducao-ao-mongodb/30792
\item https://www.mongodb.com/what-is-mongodb
\item https://www.devmedia.com.br/banco-de-dados-nosql-um-novo-paradigma-revista-sql-magazine-102/25918
\item https://aws.amazon.com/pt/nosql/columnar/
\item https://www.slideshare.net/gilesaugusto/nosql-familia-de-colunas-monografia
\item https://www.devmedia.com.br/introducao-ao-cassandra/38377
\item https://www.infoq.com/br/presentations/neo4j-visao-pratica
\item https://imasters.com.br/banco-de-dados/graphdb-series-o-que-e-um-banco-de-dados-de-grafos
\end{itemize}
\end{frame}

\begin{frame}
\frametitle{Referências Bibliográficas}
\begin{itemize}
\item https://www.slideshare.net/MayogaX/banco-de-dados-de-grafos
\item https://neo4j.com/use-cases/
\item https://www.devmedia.com.br/introducao-aos-bancos-de-dados-nosql/26044
\item https://dicasdeprogramacao.com.br/6-motivos-para-usar-bancos-de-dados-nosql/
\item http://www.cienciaedados.com/top-6-nosql-databases/
\item https://imasters.com.br/banco-de-dados/o-top-5-das-plataformas-nosql-no-mercado-atual
\item https://aws.amazon.com/pt/nosql/key-value/
\item https://db-engines.com/en/ranking
\item https://www.devmedia.com.br/introducao-ao-mongodb/30792
\item https://www.couchbase.com/resources/why-nosql
\end{itemize}
\end{frame}

\end{document}